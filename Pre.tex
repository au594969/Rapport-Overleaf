%%%%%===========================================================%%%%
%%%%% This Latex Preamble was Made by:                          %%%%
%%%%% Christian Moos Lund - 2019                                %%%%
%%%%% AU - Electrical Energy Technoligy                         %%%%
%%%%%===========================================================%%%%

%=================Kopier til main============================
%\documentclass[a4paper,oneside,article]{memoir}
%%%%%%===========================================================%%%%
%%%%% This Latex Preamble was Made by:                          %%%%
%%%%% Christian Moos Lund - 2019                                %%%%
%%%%% AU - Electrical Energy Technoligy                         %%%%
%%%%%===========================================================%%%%

%=================Kopier til main============================
%\documentclass[a4paper,oneside,article]{memoir}
%%%%%%===========================================================%%%%
%%%%% This Latex Preamble was Made by:                          %%%%
%%%%% Christian Moos Lund - 2019                                %%%%
%%%%% AU - Electrical Energy Technoligy                         %%%%
%%%%%===========================================================%%%%

%=================Kopier til main============================
%\documentclass[a4paper,oneside,article]{memoir}
%%%%%%===========================================================%%%%
%%%%% This Latex Preamble was Made by:                          %%%%
%%%%% Christian Moos Lund - 2019                                %%%%
%%%%% AU - Electrical Energy Technoligy                         %%%%
%%%%%===========================================================%%%%

%=================Kopier til main============================
%\documentclass[a4paper,oneside,article]{memoir}
%\include{Pre}
%=============================================================
% ---------------Preamble fra Latex kursus---------------------
\usepackage[utf8]{inputenc}                 % Korrekt håndtering af æ, ø og å
\usepackage[T1]{fontenc}                    % Korrekt håndtering af æ, ø og å
\usepackage{microtype}                      % Typografisk magi! Giver bl.a. pænere orddeling
\usepackage{graphicx}                       % Gør det muligt at indsætte billeder
\usepackage{amsmath}                        % Giver adgang til uundværlige matematikting
\usepackage{siunitx}                        % Dette  gør alle mat notationer nemmer ! 
\usepackage[danish]{babel}                  % Danske betegnelser og orddeling
\renewcommand{\danishhyphenmins}{22}        % Bedre dansk orddeling
%--------------------Selv fundet Pakker------------------------
\usepackage[margin=1in]{geometry}           % Marginer
\usepackage{ragged2e}                       % Retter alt ind til venstre
\usepackage{parskip}                        % Strækker ud til begge marginer
\usepackage{booktabs}                       % Tablesgenerator bad om det
\setlength{\parindent}{0pt}                 % Fjerner indents
\usepackage[bottom]{footmisc}               % Placer fodnoter i bunden af siden.
\setcounter{secnumdepth}{2}                 % Indstiller numerering af subsection (3 giver nr. til subsubsec.)
%\setcounter{tocdepth}{3}                   % Indstil til at inkluder subsection i Indholdsfortegnelse
%\usepackage{todonotes}
%--------------------Ekstra------------------------------------
\usepackage[table,xcdraw]{xcolor}
\usepackage{floatrow}
\usepackage{lscape}
\usepackage{comment}
\usepackage{multirow}
\usepackage{enumitem}                       % Enumerate med bogstaver
%---------------------For side opsætning-----------------------
\usepackage{wallpaper}
%--------------------Side opsætning----------------------------
\usepackage{tocloft}                        % Header/Footer Indholdfortegnelse
\usepackage{lastpage}                       % Til side X af Y (side nummerering)
\usepackage{fancyhdr}
\fancyhf{}                                  % sets both header and footer to nothing
\fancyhead[C]{PRJ3 EE3 \hspace{5,3cm} Gruppe 3 \hspace{4,1cm} Ingeniørhøjskolen \\ \rhead{Aarhus Universitet}}
\fancyfoot[C]{Side \thepage\ af \pageref{LastPage}}
%--------------------Referenceværktøj--------------------------
\usepackage{url}
%------------------- fysik notationer -------------------------
\usepackage{physics}
%----------------------- Macroer ------------------------------

%----------------------Tabeller--------------------------------
%\toprule               Top tyklinje
%\midrule               Mid tyndlinje
%\bottomrule            Bund tyklinje

%\begin{minipage}[t]{7CM} TEXT \vspace{1mm}\end{minipage}        Tekstombrydning

%\RaggedRight           Aligen table to the left side (replace \centering)
%\Raggedleft            Aligen table to the Right side (replace \centering)
%-------------------Andre nytte komandoer----------------------
%\plainbreak{x} x = antal linje jeg vil hoppe. 
%\graphicspath{ {./images/} }
%\begin{landscape} og \end{landscape}

%-------------------Side-by-side figur-------------------------
%\begin{figure}[H]
%\begin{floatrow}
%\ffigbox{\includegraphics[scale = 1]{}}{\caption{}
%\label{fig:Pkontrolergraf}}
%\ffigbox{\includegraphics[scale = 1]{}}{\caption{}
%\label{fig:}}
%\end{floatrow}
%\end{figure}


%%%%=========================Tegn tæller prototype======================
%%%%-------------------Inde i doc---------------------------------------
%%%\quickcharcount{main} 
%%%%---------------------Ignor code-------------------------------------
%%%---%%TC:ignore og %%TC:endignore
%%%%----------------------Tegn macro------------------------------------
%%%\newread\tmp
%%%
%%%\newcommand{\quickcharcount}[1]{%
%%%    \immediate\write18{texcount -1 -sum -merge -char #1.tex > #1-chars.sum}%
%%%    \openin\tmp=#1-chars.sum%
%%%    \read\tmp to \thechar%
%%%    \closein\tmp%
%%%} <----Husk
%%%
%%%\newcommand{\quickwordcount}[1]{%
%%%    \immediate\write18{texcount -1 -sum -merge #1.tex > #1-words.sum}%
%%%    \openin\tmp=#1-words.sum%
%%%    \read\tmp to \theword%
%%%    \closein\tmp%
%%%} <----Husk
%%%%----------------------------------------------------------------------

%=============================================================
% ---------------Preamble fra Latex kursus---------------------
\usepackage[utf8]{inputenc}                 % Korrekt håndtering af æ, ø og å
\usepackage[T1]{fontenc}                    % Korrekt håndtering af æ, ø og å
\usepackage{microtype}                      % Typografisk magi! Giver bl.a. pænere orddeling
\usepackage{graphicx}                       % Gør det muligt at indsætte billeder
\usepackage{amsmath}                        % Giver adgang til uundværlige matematikting
\usepackage{siunitx}                        % Dette  gør alle mat notationer nemmer ! 
\usepackage[danish]{babel}                  % Danske betegnelser og orddeling
\renewcommand{\danishhyphenmins}{22}        % Bedre dansk orddeling
%--------------------Selv fundet Pakker------------------------
\usepackage[margin=1in]{geometry}           % Marginer
\usepackage{ragged2e}                       % Retter alt ind til venstre
\usepackage{parskip}                        % Strækker ud til begge marginer
\usepackage{booktabs}                       % Tablesgenerator bad om det
\setlength{\parindent}{0pt}                 % Fjerner indents
\usepackage[bottom]{footmisc}               % Placer fodnoter i bunden af siden.
\setcounter{secnumdepth}{2}                 % Indstiller numerering af subsection (3 giver nr. til subsubsec.)
%\setcounter{tocdepth}{3}                   % Indstil til at inkluder subsection i Indholdsfortegnelse
%\usepackage{todonotes}
%--------------------Ekstra------------------------------------
\usepackage[table,xcdraw]{xcolor}
\usepackage{floatrow}
\usepackage{lscape}
\usepackage{comment}
\usepackage{multirow}
\usepackage{enumitem}                       % Enumerate med bogstaver
%---------------------For side opsætning-----------------------
\usepackage{wallpaper}
%--------------------Side opsætning----------------------------
\usepackage{tocloft}                        % Header/Footer Indholdfortegnelse
\usepackage{lastpage}                       % Til side X af Y (side nummerering)
\usepackage{fancyhdr}
\fancyhf{}                                  % sets both header and footer to nothing
\fancyhead[C]{PRJ3 EE3 \hspace{5,3cm} Gruppe 3 \hspace{4,1cm} Ingeniørhøjskolen \\ \rhead{Aarhus Universitet}}
\fancyfoot[C]{Side \thepage\ af \pageref{LastPage}}
%--------------------Referenceværktøj--------------------------
\usepackage{url}
%------------------- fysik notationer -------------------------
\usepackage{physics}
%----------------------- Macroer ------------------------------

%----------------------Tabeller--------------------------------
%\toprule               Top tyklinje
%\midrule               Mid tyndlinje
%\bottomrule            Bund tyklinje

%\begin{minipage}[t]{7CM} TEXT \vspace{1mm}\end{minipage}        Tekstombrydning

%\RaggedRight           Aligen table to the left side (replace \centering)
%\Raggedleft            Aligen table to the Right side (replace \centering)
%-------------------Andre nytte komandoer----------------------
%\plainbreak{x} x = antal linje jeg vil hoppe. 
%\graphicspath{ {./images/} }
%\begin{landscape} og \end{landscape}

%-------------------Side-by-side figur-------------------------
%\begin{figure}[H]
%\begin{floatrow}
%\ffigbox{\includegraphics[scale = 1]{}}{\caption{}
%\label{fig:Pkontrolergraf}}
%\ffigbox{\includegraphics[scale = 1]{}}{\caption{}
%\label{fig:}}
%\end{floatrow}
%\end{figure}


%%%%=========================Tegn tæller prototype======================
%%%%-------------------Inde i doc---------------------------------------
%%%\quickcharcount{main} 
%%%%---------------------Ignor code-------------------------------------
%%%---%%TC:ignore og %%TC:endignore
%%%%----------------------Tegn macro------------------------------------
%%%\newread\tmp
%%%
%%%\newcommand{\quickcharcount}[1]{%
%%%    \immediate\write18{texcount -1 -sum -merge -char #1.tex > #1-chars.sum}%
%%%    \openin\tmp=#1-chars.sum%
%%%    \read\tmp to \thechar%
%%%    \closein\tmp%
%%%} <----Husk
%%%
%%%\newcommand{\quickwordcount}[1]{%
%%%    \immediate\write18{texcount -1 -sum -merge #1.tex > #1-words.sum}%
%%%    \openin\tmp=#1-words.sum%
%%%    \read\tmp to \theword%
%%%    \closein\tmp%
%%%} <----Husk
%%%%----------------------------------------------------------------------

%=============================================================
% ---------------Preamble fra Latex kursus---------------------
\usepackage[utf8]{inputenc}                 % Korrekt håndtering af æ, ø og å
\usepackage[T1]{fontenc}                    % Korrekt håndtering af æ, ø og å
\usepackage{microtype}                      % Typografisk magi! Giver bl.a. pænere orddeling
\usepackage{graphicx}                       % Gør det muligt at indsætte billeder
\usepackage{amsmath}                        % Giver adgang til uundværlige matematikting
\usepackage{siunitx}                        % Dette  gør alle mat notationer nemmer ! 
\usepackage[danish]{babel}                  % Danske betegnelser og orddeling
\renewcommand{\danishhyphenmins}{22}        % Bedre dansk orddeling
%--------------------Selv fundet Pakker------------------------
\usepackage[margin=1in]{geometry}           % Marginer
\usepackage{ragged2e}                       % Retter alt ind til venstre
\usepackage{parskip}                        % Strækker ud til begge marginer
\usepackage{booktabs}                       % Tablesgenerator bad om det
\setlength{\parindent}{0pt}                 % Fjerner indents
\usepackage[bottom]{footmisc}               % Placer fodnoter i bunden af siden.
\setcounter{secnumdepth}{2}                 % Indstiller numerering af subsection (3 giver nr. til subsubsec.)
%\setcounter{tocdepth}{3}                   % Indstil til at inkluder subsection i Indholdsfortegnelse
%\usepackage{todonotes}
%--------------------Ekstra------------------------------------
\usepackage[table,xcdraw]{xcolor}
\usepackage{floatrow}
\usepackage{lscape}
\usepackage{comment}
\usepackage{multirow}
\usepackage{enumitem}                       % Enumerate med bogstaver
%---------------------For side opsætning-----------------------
\usepackage{wallpaper}
%--------------------Side opsætning----------------------------
\usepackage{tocloft}                        % Header/Footer Indholdfortegnelse
\usepackage{lastpage}                       % Til side X af Y (side nummerering)
\usepackage{fancyhdr}
\fancyhf{}                                  % sets both header and footer to nothing
\fancyhead[C]{PRJ3 EE3 \hspace{5,3cm} Gruppe 3 \hspace{4,1cm} Ingeniørhøjskolen \\ \rhead{Aarhus Universitet}}
\fancyfoot[C]{Side \thepage\ af \pageref{LastPage}}
%--------------------Referenceværktøj--------------------------
\usepackage{url}
%------------------- fysik notationer -------------------------
\usepackage{physics}
%----------------------- Macroer ------------------------------

%----------------------Tabeller--------------------------------
%\toprule               Top tyklinje
%\midrule               Mid tyndlinje
%\bottomrule            Bund tyklinje

%\begin{minipage}[t]{7CM} TEXT \vspace{1mm}\end{minipage}        Tekstombrydning

%\RaggedRight           Aligen table to the left side (replace \centering)
%\Raggedleft            Aligen table to the Right side (replace \centering)
%-------------------Andre nytte komandoer----------------------
%\plainbreak{x} x = antal linje jeg vil hoppe. 
%\graphicspath{ {./images/} }
%\begin{landscape} og \end{landscape}

%-------------------Side-by-side figur-------------------------
%\begin{figure}[H]
%\begin{floatrow}
%\ffigbox{\includegraphics[scale = 1]{}}{\caption{}
%\label{fig:Pkontrolergraf}}
%\ffigbox{\includegraphics[scale = 1]{}}{\caption{}
%\label{fig:}}
%\end{floatrow}
%\end{figure}


%%%%=========================Tegn tæller prototype======================
%%%%-------------------Inde i doc---------------------------------------
%%%\quickcharcount{main} 
%%%%---------------------Ignor code-------------------------------------
%%%---%%TC:ignore og %%TC:endignore
%%%%----------------------Tegn macro------------------------------------
%%%\newread\tmp
%%%
%%%\newcommand{\quickcharcount}[1]{%
%%%    \immediate\write18{texcount -1 -sum -merge -char #1.tex > #1-chars.sum}%
%%%    \openin\tmp=#1-chars.sum%
%%%    \read\tmp to \thechar%
%%%    \closein\tmp%
%%%} <----Husk
%%%
%%%\newcommand{\quickwordcount}[1]{%
%%%    \immediate\write18{texcount -1 -sum -merge #1.tex > #1-words.sum}%
%%%    \openin\tmp=#1-words.sum%
%%%    \read\tmp to \theword%
%%%    \closein\tmp%
%%%} <----Husk
%%%%----------------------------------------------------------------------

%=============================================================
% ---------------Preamble fra Latex kursus---------------------
\usepackage[utf8]{inputenc}                 % Korrekt håndtering af æ, ø og å
\usepackage[T1]{fontenc}                    % Korrekt håndtering af æ, ø og å
\usepackage{microtype}                      % Typografisk magi! Giver bl.a. pænere orddeling
\usepackage{graphicx}                       % Gør det muligt at indsætte billeder
\usepackage{amsmath}                        % Giver adgang til uundværlige matematikting
\usepackage{siunitx}                        % Dette  gør alle mat notationer nemmer ! 
\usepackage[danish]{babel}                  % Danske betegnelser og orddeling
\renewcommand{\danishhyphenmins}{22}        % Bedre dansk orddeling
%--------------------Selv fundet Pakker------------------------
\usepackage[margin=1in]{geometry}           % Marginer
\usepackage{ragged2e}                       % Retter alt ind til venstre
\usepackage{parskip}                        % Strækker ud til begge marginer
\usepackage{booktabs}                       % Tablesgenerator bad om det
\setlength{\parindent}{0pt}                 % Fjerner indents
\usepackage[bottom]{footmisc}               % Placer fodnoter i bunden af siden.
\setcounter{secnumdepth}{2}                 % Indstiller numerering af subsection (3 giver nr. til subsubsec.)
%\setcounter{tocdepth}{3}                   % Indstil til at inkluder subsection i Indholdsfortegnelse
%\usepackage{todonotes}
%--------------------Ekstra------------------------------------
\usepackage[table,xcdraw]{xcolor}
\usepackage{floatrow}
\usepackage{lscape}
\usepackage{comment}
\usepackage{multirow}
\usepackage{enumitem}                       % Enumerate med bogstaver
%---------------------For side opsætning-----------------------
\usepackage{wallpaper}
%--------------------Side opsætning----------------------------
\usepackage{tocloft}                        % Header/Footer Indholdfortegnelse
\usepackage{lastpage}                       % Til side X af Y (side nummerering)
\usepackage{fancyhdr}
\fancyhf{}                                  % sets both header and footer to nothing
\fancyhead[C]{PRJ3 EE3 \hspace{5,3cm} Gruppe 3 \hspace{4,1cm} Ingeniørhøjskolen \\ \rhead{Aarhus Universitet}}
\fancyfoot[C]{Side \thepage\ af \pageref{LastPage}}
%--------------------Referenceværktøj--------------------------
\usepackage{url}
%------------------- fysik notationer -------------------------
\usepackage{physics}
%----------------------- Macroer ------------------------------

%----------------------Tabeller--------------------------------
%\toprule               Top tyklinje
%\midrule               Mid tyndlinje
%\bottomrule            Bund tyklinje

%\begin{minipage}[t]{7CM} TEXT \vspace{1mm}\end{minipage}        Tekstombrydning

%\RaggedRight           Aligen table to the left side (replace \centering)
%\Raggedleft            Aligen table to the Right side (replace \centering)
%-------------------Andre nytte komandoer----------------------
%\plainbreak{x} x = antal linje jeg vil hoppe. 
%\graphicspath{ {./images/} }
%\begin{landscape} og \end{landscape}

%-------------------Side-by-side figur-------------------------
%\begin{figure}[H]
%\begin{floatrow}
%\ffigbox{\includegraphics[scale = 1]{}}{\caption{}
%\label{fig:Pkontrolergraf}}
%\ffigbox{\includegraphics[scale = 1]{}}{\caption{}
%\label{fig:}}
%\end{floatrow}
%\end{figure}


%%%%=========================Tegn tæller prototype======================
%%%%-------------------Inde i doc---------------------------------------
%%%\quickcharcount{main} 
%%%%---------------------Ignor code-------------------------------------
%%%---%%TC:ignore og %%TC:endignore
%%%%----------------------Tegn macro------------------------------------
%%%\newread\tmp
%%%
%%%\newcommand{\quickcharcount}[1]{%
%%%    \immediate\write18{texcount -1 -sum -merge -char #1.tex > #1-chars.sum}%
%%%    \openin\tmp=#1-chars.sum%
%%%    \read\tmp to \thechar%
%%%    \closein\tmp%
%%%} <----Husk
%%%
%%%\newcommand{\quickwordcount}[1]{%
%%%    \immediate\write18{texcount -1 -sum -merge #1.tex > #1-words.sum}%
%%%    \openin\tmp=#1-words.sum%
%%%    \read\tmp to \theword%
%%%    \closein\tmp%
%%%} <----Husk
%%%%----------------------------------------------------------------------
